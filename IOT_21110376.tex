\documentclass{article}
\usepackage[utf8]{inputenc}

\title{IOT CIA 21110376}
\author{Cynddia Balamurugan}

\begin{document}

\maketitle
\begin{center}
    \textbf{PAPER REVIEW}
    \\{A Study on IoT Technologies, Standards and Protocols
\\ Author: Karthik Kumar Vaigandla, Krishna Radha
\\Conference: IBMRD's Journal of Management & ResearchVolume: 10
\\Link:https://www.researchgate.net/publication/359203501_A_S
tudy_on_IoT_Technologies_Standards_and_Protocols
}

\end{center}
\section{Introduction}
\\ 

IoT( internet of things) is a term that describes a recent improvement in connecting  gadgets People and things are practically connected as a result of IoT technology. IOT enables seamless contact between social media and the internet and facilitates the development of new services and applications.

 The two essential components of IoT are "things" and the internet. Utilizing intelligence and consensus, technology enables things to make decisions that are authoritative and advantageous to many applications. These pose difficulties that need certain communication guidelines.

The Internet of Things paradigm is supported by a number of current technologies, including as Bluetooth, ZigBee, Wi-Fi, and Long Term Evolution Advanced (LTE-A). The development of an IoT system that uses these technologies as its foundation will probably be quite difficult. IoT scalability and increased functionality are essential, and this system also needs sensor identifiers. To to be ecologically sustainable, they must be energy-efficient and have effective data management mechanisms.

The majority of IoT-connected devices have a radio that is underpowered, little memory, bad batteries, and limited computing power. The TCP/IP stack is incompatible with this environment, hence working groups have started modifying current protocols to new IoT updates [4]. An addressing method like IPv6 is crucial since there are so many interconnected nodes. Many working groups, including LoWPAN, IEEE, and ZigBee, are already proposing approaches to enable IPv6 in limited contexts. Attacks using a denial of service are increasing.

\\

\section{Internet of Things(IOT)}
\\
The IoT has a wide range of components, allowing data to be transferred without the need for human or computer intervention. In the IoT, there are three types of things: sensors that gather data and communicate it to a server, computers that take information in and do something with it, and Things that do both. Since they may be linked with computer-based systems at a lower point of integration, objects can be sensed and controlled remotely through existing network infrastructure. This reduces the need for human interaction while increasing efficiency, accuracy, and economic value.

Since the gadgets are in direct contact, they can instantly share information without a mediator. Device-to-gateway connections are used for communication between sensors and gateway nodes. Through a connection between a gateway and a data system, data is sent from one to the other.

The main applications of IOT are of the applications are wearables, health monitoring,
traffic monitoring, fleet management, agriculture, Hospitality, water supply, maintenance management, industrial automation, smart grid and energy saving.

\\

\section{ Technologies and protocols for the Internet of Things}
\\
\\
\subsection{Bluetooth and Bluetooth Low Energy (BLE)}
\\
 Bluetooth is a popular short-range communication protocol. This protocol is used to wirelessly communicate IoT data.The Bluetooth protocol is inexpensive, short-range, and low-power compared to other wireless protocols. Low-energy versions of Bluetooth are available thanks to the Bluetooth low energy (BLE) protocol.
\\
\subsection{ZigBee}
\\
 The ZigBee protocol allows smart things to talk to one another. ZigBee can carry data over short  distances. Most ZigBee networks are found in industrial environments. The most recent version of ZigBee, v3.0, combines multiple ZigBee standards into a single one. A set of ZigBee protocol requirements for remote control of compact, low-power radios are defined under the IEEE 802.15.4-2003 standard. 
\\
\subsection{IP ZigBee}
\\
The first open standard is an IPv6-based full-mesh wireless network based on ZigBee IP. Without compromising on power or cost, the gadget enables simple control of thousands of devices to offer seamless Internet connectivity. The ZigBee Smart Energy Internet Protocol system is supported by a ZigBee internet protocol. Figure 7 shows a ZigBee IP stack that is based on the low layer IEEE 802.15.4 standard. All nodes of the network can communicate with each other via the IPv6 addressing and routing protocol.
\\
\subsection{ ZigBee IP}
\\
The first open standard is an IPv6-based full-mesh wireless network based on ZigBee IP. Without compromising on power or cost, the gadget enables simple control of thousands of devices to offer seamless Internet connectivity. The ZigBee Smart Energy Internet is supported by a ZigBee internet protocol.
\\
\subsection{Long Range Wide Area Network (LoRaWAN)}
\\
LoRaWAN is a wireless technology with benefits including cheap cost, low power consumption, mobility, security, and bidirectional communication that is made for Internet of Things applications. This technique can identify low-strength signals across great distances when the noise level is low. This protocol is powered by low power consumption and has been tuned for scalable networks with millions of wireless devices. It enables redundant operation, low cost, low power, and energy harvesting technologies that enable simplicity of use and mobility in order to satisfy the needs of the IoT in the future [3]. It is a protocol designed to operate with Media Access Control (MAC) to handle massive public networks with a single operator.
\\
\subsection{LoWPAN}
\\
 One of the most important Internet of Things protocols is LoWPAN.Sensors and small IoT devices may safely and securely connect with one another.For embedded devices and interconnected networks, LowPAN was created by an IETF working group and published in RFC4944. In order to support wireless IPv6 data transmissions, 6LoWPAN later developed encapsulation and compression.

A 6LoWPAN packet's type can be determined by its first two bits. Type and the following 6 bits determine the rest of the structure. A no 6LoWPAN header (00), a dispatch header (01), a mesh header (10), and a fragmentation header are the four types of headers used by 6LoWPAN frames based on two bits (11)
\\
\subsection{LTE Advanced (LTE-A)}
\\
The Long Term Evolution (LTE) network standard, which represents the newest 4G network technology, was created in 2008. LTE-A (advanced) enhances the architecture of LTE. This entails improving network capacity, power efficiency, spectrum efficiency, and operator cost reduction [2]. Since it was first announced in 2009, LTE-A has undergone various changes to suit new technology.
\\
\subsection{Z-Wave}
\\
In the wireless Z-Wave technology, low energy radio waves are used. The system is largely used to operate wirelessly connected household equipment including lighting, security, thermostats, garage door openers, etc. Z-Wave can connect objects up to 30 metres away using mesh technology. Small ACK messages and CSMA/CA are employed for reliable transmission.
\\
\subsection{CORPL, RPL, and RPL Enhancements}
\\
The IETF released a brand-new protocol in 2012  referred to as Distance Vector Routing Protocol for Low Power and Lossy Networks (RPL). When using RPL, a Destination Oriented Directed Acyclic Graph, there is only one way to get from any leaf node to any root node (DODAG). Initial broadcasts from each node via the DODAG information object identify them as the root (DIO). A parent sends an advertisement (DAO) to the root when a node connects with its parents, and the root determines where to route it. The aim of Enhanced-RPL is to improve the RPL protocol's reliability. Dynamic RPL is a useful tool for IoT applications that use dynamic logic.
\\
\subsection{E-CARP and CARP}
\\
Underwater Wireless Sensor Networks (UWSNs) utilise the non-standard distributed routing protocol Channel. Routing Protocol Aware (CARP). This method uses less energy and delivers packets in a fair amount of time. The standard of the historical links is measured and taken into consideration while choosing the forwarding path. The history from nearby sensors is acquired to choose the routing nodes. The drawback of CARP is that previously obtained data cannot be used again. The sink node can store previously acquired sensory data thanks to enhanced-CARP. When new data is needed, sensor nodes react to E-CARP packets with it.With E-CARP, communication overhead is significantly decreased.
\\
\subsection{Message Queue Telemetry Transport}
\\
The messaging protocol Messaging Queue Telemetry Transport, links embedded devices with software and middleware. It is typically implemented over TCP due to its minimal resource needs. There are three parts to this entity: subscribers, publishers, and brokers. A broker acts as a security by giving both parties permission. Due to its minimal resource utilisation, it is frequently used in monitoring, Facebook notifications, health care, and machine-to-machine messaging. Message headers range from one to four bytes in size, depending on the message's length.
\\
\subsection{Constrained Application Protocol (CoAP)}
\\
The CoRE (Constrained Resource Environments) group created the IETF standard known as Constrained Application Protocol. Similar to HTTP, CoAP has a client-server interaction architecture. CoAP solutions that can serve as both clients and servers are typically used in machine-to-machine communication. Simple, restricted devices can access the IoT using CoAP even in constrained circumstances through low bandwidth, poor availability networks. Representational state transfer is a common interface used in current online applications (REST). Since UDP rather than TCP is employed in its construction, a lightweight technique is required to maintain reliability.
\\
\section{Conclusion}
\\
IOT aims to enhance living quality by automating, integrating hardware and software, and increasing information accessibility and speed. It covers a few common technologies created especially for embedded systems and other constrained contexts.At each level, the majority of the standards that had been finalised were presented, and some prototypes were highlighted.

 It is hard to prioritise one IoT protocol above another in terms of Internet of things technology because they all have unique applicability in unique situations. We can establish a strong networking foundation for the Internet of Things of the future by developing and updating the technical foundation. Scholars and professionals can uncover holes in network typologies, create more effective protocols, and address

\\
\section{Agreements, Fitfalls and Fallacies:}
\\
\subsection{Agreements:}
    The IoT is increasingly used in industries including medicine, engineering, safety, and transportation.
    \\
    Devices connected to the Internet of Things generally have low memory, inadequate batteries, limited processing capabilities, and a weak radio.
    \\
    By automating, connecting devices and applications and making information faster and more available, it looks to improve life quality.   
\\

\subsection{Disgreements:}
\\
Compared to other alternatives, Z-Wave uses a simpler protocol, which makes development easier. 
Zigbee protocol is better for using less energy than the Z-Wave protocol. Zigbee protocol was built on the IEEE 802.15. 4 protocol standard.
\\


\end{document}
